%%% Circuit-Example.tex
%%% version 1.0, July 10, 2017
%%%
%%% A minimal example for typesetting proof circuits using LaTeX and tikz.
%%%
%%% Greg Restall / greg@consequently.org 
%%% 
%%% Depedencies: 
%%% tikz, available at https://ctan.org/pkg/pgf
%%% ebproof, avaialble at https://ctan.org/pkg/ebproof
%%%
%%% Questions? Contact Greg Restall  / http://consequently.org
%%%
\documentclass{article}
%%
\usepackage{fancyvrb}
\RecustomVerbatimEnvironment{Verbatim}{Verbatim}{frame=lines,fontsize=\small,framesep=1.5ex}
\usepackage[colorlinks,urlcolor=Blue]{hyperref}
%%
\usepackage{tikz}
\definecolor{Red}{rgb}{0.66,0,0}
\definecolor{Green}{rgb}{0,0.5,0}
\definecolor{Blue}{rgb}{0,0,0.66}
%%%
%%% \tm{atom}{label} typesets "atom" in math mode, and 
%%% assigns a tikz node "node" which can later be used in
%%% a tikz diagram.
%%%
\newcommand{\tm}[2]{%
  \ensuremath{\mathord{%
  \tikz[remember picture,baseline=(#2.base)]%
  \node[inner sep=.5pt,outer sep=.5pt](#2){\({#1}\)};%
}}%
}%
%%%
%%% flowgraph is a tikz environment for drawing 
%%% linkages between nodes defined by "\tm" entries.
%%%
%%% The default arrow style is Red, thick, and with a stealth arrowhead 
%%%
\tikzstyle{flowgraph} = [overlay, remember picture,
                         Red, thick, ->,
                         >=stealth, shorten >=0.5pt]
%%%
\usepackage{ebproof}
%%%
%%%
\begin{document}
\title{Typesetting Flow Graphs with \texttt{tikz}}
\author{Greg Restall}
\maketitle
\noindent%
These notes attempt to explain how to use the \LaTeX\ package \texttt{tikz} to typeset flow graphs, used for indicating the flow of information in different kinds of proofs. The code can be downloaded from \url{https://github.com/consequently/flowgraphs}.

In your \LaTeX\ document, to set things up, load \texttt{tikz} in the preamble and define some colours.
\begin{Verbatim}
\usepackage{tikz}
\definecolor{Red}{rgb}{0.66,0,0}
\definecolor{Green}{rgb}{0,0.5,0}
\definecolor{Blue}{rgb}{0,0,0.66}
\end{Verbatim}
In this document, I will typeset some proofs and derivations. For that, I'll use \texttt{ebproof}, so I need to load that package too:
\begin{Verbatim}
\usepackage{ebproof}
\end{Verbatim}
Both \texttt{\href{https://ctan.org/pkg/pgf}{tikz}} and \texttt{\href{https://ctan.org/pkg/ebproof}{ebproof}} are available on \texttt{\href{http://ctan.org}{ctan}}, and they should be already on any reasonably up-to-date \LaTeX\ installation. 

To typeset flow graphs with \texttt{tikz}, you use two definitions. The first, \Verb|\tm| (short for ``a\textbf{t}o\textbf{m}'') defines the nodes for the graph:
\begin{Verbatim}
\newcommand{\tm}[2]{%
  \ensuremath{\mathord{%
  \tikz[remember picture,baseline=(#2.base)]%
  \node[inner sep=.5pt,outer sep=.5pt](#2){\({#1}\)};%
}}%
}%
\end{Verbatim}
\Verb|\tm{p}{p1}| typesets $p$ in math mode, inside a tikz picture, as a node with label \Verb|p1|, which can be used as the source or a target for a link in a graph, typset later in the code.  To typset the links, you use a \texttt{tikzpicture} with the \Verb|flowgraph| style:  
\begin{Verbatim}
\tikzstyle{flowgraph} = [overlay, remember picture,
                         Red, thick, ->,
                         >=stealth, shorten >=0.5pt]
\end{Verbatim}
Such a \texttt{tikzpicture} is typeset as an \emph{overlay} (it is typeset over the other text), and the default links are set in the color \texttt{Red}, with the \texttt{tikz} style \texttt{thick}, as arrows with the ``\texttt{stealth}'' arrowhead, and shortened at the end by a little bit (for clarity). Of course, these defaults can all be overridden. (In the examples below you'll see \texttt{Blue} and \texttt{Green} links, too.)
\section*{The General Idea}
You can mark an atom like this \tm{p}{p1}, typesetting it in a small tikz picture, where you remember its node name and location, and do the same with another one like this---\tm{p}{p2}---and then typeset a link from the one to the other. 
\begin{tikzpicture}[flowgraph]
\draw (p1) -- (p2); 
\end{tikzpicture}
\begin{Verbatim}
You can mark an atom like this \tm{p}{p1}, typesetting it in a small 
tikz picture, where you remember its node name and location, and do the 
same with another one like this---\tm{p}{p2}---and then typeset a link 
from the one to the other. 
\begin{tikzpicture}[flowgraph]
\draw (p1) -- (p2); 
\end{tikzpicture}
\end{Verbatim}
Here are some concrete examples of using flow graphs in proofs.

\section*{Example 1: In an Array}
\[
\begin{array}{c}
\tm{p}{pp}\land(\tm{q}{qp}\lor \tm{r}{rp})\\[0.66cm]
(\tm{p}{p1c}\land \tm{q}{qc})\lor(\tm{p}{p2c}\land \tm{r}{rc})
\end{array}
\]
\begin{tikzpicture}[flowgraph]
\draw (pp) -- (p1c); \draw (pp) -- (p2c);
\draw (qp) -- (qc);  \draw (rp) -- (rc);
\end{tikzpicture}
%%
\begin{Verbatim}
\[
\begin{array}{c}
\tm{p}{pp}\land(\tm{q}{qp}\lor \tm{r}{rp}) \\[0.66cm]
(\tm{p}{p1c}\land \tm{q}{qc})\lor(\tm{p}{p2c}\land \tm{r}{rc})
\end{array}
\]
\begin{tikzpicture}[flowgraph]
\draw (pp) -- (p1c); \draw (pp) -- (p2c);
\draw (qp) -- (qc);  \draw (rp) -- (rc);
\end{tikzpicture}
\end{Verbatim}
\section*{Example 2: In a Natural Deduction Proof}
% Vertical space to give room for looping links
\vspace{0.5cm} 
\[
\begin{prooftree}[rule margin=.9ex, separation=0.2em]
\Hypo{p\land(\tm{q}{q1}\lor \tm{r}{r1})}
\Infer1{\tm{q}{q2}\lor \tm{r}{r2}}
\Hypo{\tm{p}{p1}\land(q\lor r)}
\Infer1{\tm{p}{p3}}
\Hypo{[\tm{q}{q3}]}
\Infer2{\tm{p}{p5}\land \tm{q}{q4}}
\Infer1{(\tm{p}{p7}\land \tm{q}{q5})\lor(\tm{p}{p8}\land \tm{r}{r5})}
\Hypo{\tm{p}{p2}\land(q\lor r)}
\Infer1{\tm{p}{p4}}
\Hypo{[\tm{r}{r3}]}
\Infer2{\tm{p}{p6}\land \tm{r}{r4}}
\Infer1{(\tm{p}{p9}\land \tm{q}{q6})\lor(\tm{p}{p10}\land \tm{r}{r6})}
\Infer3{(\tm{p}{p11}\land \tm{q}{q7})\lor(\tm{p}{p12}\land \tm{r}{r7})}
\end{prooftree}
\]
\begin{tikzpicture}[flowgraph,bend angle=90]
\draw (p1) -- (p3);  \draw (p3) -- (p5); 
\draw (p5) -- (p7);  \draw (p7) -- (p11);
\draw (p2) -- (p4);  \draw (p4) -- (p6);
\draw (p6) -- (p10); \draw (p8) -- (p12);
\draw (p9) -- (p11); \draw (p10) -- (p12);
%%
\begin{scope}[Green]
\draw (r1) -- (r2); 
\path (r2) edge[bend left] (r3);
\draw (r3) -- (r4); \draw (r4) -- (r6);
\draw (r5) -- (r7); \draw (r6) -- (r7);
\end{scope}
%%
\begin{scope}[Blue]
\draw (q1) -- (q2);
\path (q2) edge[bend left] (q3);
\draw (q3) -- (q4); \draw (q4) -- (q5);
\draw (q5) -- (q7); \draw (q6) -- (q7);
\end{scope}
\end{tikzpicture}%
Using different colours can help with distinguishing different paths.
\begin{Verbatim}
% Vertical space to give room for looping links
\vspace{0.5cm} 
\[
\begin{prooftree}[rule margin=.9ex, separation=0.2em]
\Hypo{p\land(\tm{q}{q1}\lor \tm{r}{r1})}
\Infer1{\tm{q}{q2}\lor \tm{r}{r2}}
\Hypo{\tm{p}{p1}\land(q\lor r)}
\Infer1{\tm{p}{p3}}
\Hypo{[\tm{q}{q3}]}
\Infer2{\tm{p}{p5}\land \tm{q}{q4}}
\Infer1{(\tm{p}{p7}\land \tm{q}{q5})\lor(\tm{p}{p8}\land \tm{r}{r5})}
\Hypo{\tm{p}{p2}\land(q\lor r)}
\Infer1{\tm{p}{p4}}
\Hypo{[\tm{r}{r3}]}
\Infer2{\tm{p}{p6}\land \tm{r}{r4}}
\Infer1{(\tm{p}{p9}\land \tm{q}{q6})\lor(\tm{p}{p10}\land \tm{r}{r6})}
\Infer3{(\tm{p}{p11}\land \tm{q}{q7})\lor(\tm{p}{p12}\land \tm{r}{r7})}
\end{prooftree}
\]
%% p links
\begin{tikzpicture}[flowgraph,bend angle=90]
\draw (p1) -- (p3);  \draw (p3) -- (p5); 
\draw (p5) -- (p7);  \draw (p7) -- (p11);
\draw (p2) -- (p4);  \draw (p4) -- (p6);
\draw (p6) -- (p10); \draw (p8) -- (p12);
\draw (p9) -- (p11); \draw (p10) -- (p12);
%% q links
\begin{scope}[Blue]
\draw (q1) -- (q2);
\path (q2) edge[bend left] (q3);
\draw (q3) -- (q4); \draw (q4) -- (q5);
\draw (q5) -- (q7); \draw (q6) -- (q7);
\end{scope}
%% r links
\begin{scope}[Green]
\draw (r1) -- (r2); 
\path (r2) edge[bend left] (r3);
\draw (r3) -- (r4); \draw (r4) -- (r6);
\draw (r5) -- (r7); \draw (r6) -- (r7);
\end{scope}
\end{tikzpicture}
\end{Verbatim}
However, there it is still difficult to follow such a complex flow graph with many different links at play. You can reuse the same derivation code and link up only \emph{some} of the atoms. The new node labels take precedence over the old ones.
\[
\begin{prooftree}[rule margin=.9ex, separation=0.2em]
\Hypo{p\land(\tm{q}{q1}\lor \tm{r}{r1})}
\Infer1{\tm{q}{q2}\lor \tm{r}{r2}}
\Hypo{\tm{p}{p1}\land(q\lor r)}
\Infer1{\tm{p}{p3}}
\Hypo{[\tm{q}{q3}]}
\Infer2{\tm{p}{p5}\land \tm{q}{q4}}
\Infer1{(\tm{p}{p7}\land \tm{q}{q5})\lor(\tm{p}{p8}\land \tm{r}{r5})}
\Hypo{\tm{p}{p2}\land(q\lor r)}
\Infer1{\tm{p}{p4}}
\Hypo{[\tm{r}{r3}]}
\Infer2{\tm{p}{p6}\land \tm{r}{r4}}
\Infer1{(\tm{p}{p9}\land \tm{q}{q6})\lor(\tm{p}{p10}\land \tm{r}{r6})}
\Infer3{(\tm{p}{p11}\land \tm{q}{q7})\lor(\tm{p}{p12}\land \tm{r}{r7})}
\end{prooftree}
\]
Here's the code:
\begin{tikzpicture}[flowgraph,bend angle=90]
\draw (p1) -- (p3);  \draw (p3) -- (p5); 
\draw (p5) -- (p7);  \draw (p7) -- (p11);
\draw (p2) -- (p4);  \draw (p4) -- (p6);
\draw (p6) -- (p10); \draw (p8) -- (p12);
\draw (p9) -- (p11); \draw (p10) -- (p12);
\end{tikzpicture}
\begin{Verbatim}
\[
\begin{prooftree}[rule margin=.9ex, separation=0.2em]
\Hypo{p\land(\tm{q}{q1}\lor \tm{r}{r1})}
\Infer1{\tm{q}{q2}\lor \tm{r}{r2}}
\Hypo{\tm{p}{p1}\land(q\lor r)}
\Infer1{\tm{p}{p3}}
\Hypo{[\tm{q}{q3}]}
\Infer2{\tm{p}{p5}\land \tm{q}{q4}}
\Infer1{(\tm{p}{p7}\land \tm{q}{q5})\lor(\tm{p}{p8}\land \tm{r}{r5})}
\Hypo{\tm{p}{p2}\land(q\lor r)}
\Infer1{\tm{p}{p4}}
\Hypo{[\tm{r}{r3}]}
\Infer2{\tm{p}{p6}\land \tm{r}{r4}}
\Infer1{(\tm{p}{p9}\land \tm{q}{q6})\lor(\tm{p}{p10}\land \tm{r}{r6})}
\Infer3{(\tm{p}{p11}\land \tm{q}{q7})\lor(\tm{p}{p12}\land \tm{r}{r7})}
\end{prooftree}
\]
\begin{tikzpicture}[flowgraph,bend angle=90]
\draw (p1) -- (p3);  \draw (p3) -- (p5); 
\draw (p5) -- (p7);  \draw (p7) -- (p11);
\draw (p2) -- (p4);  \draw (p4) -- (p6);
\draw (p6) -- (p10); \draw (p8) -- (p12);
\draw (p9) -- (p11); \draw (p10) -- (p12);
\end{tikzpicture}
\end{Verbatim}
\section*{Example 3: In a Sequent Derivation}
\[
\begin{prooftree}
\Hypo{\tm{p}{p1}\vdash \tm{p}{p2}}
\Hypo{\tm{q}{q1}\vdash \tm{q}{q2}}
\Infer2{\tm{p}{p5},\tm{q}{q3}\vdash \tm{p}{p6}\land \tm{q}{q4}}
\Hypo{\tm{p}{p3}\vdash \tm{p}{p4}}
\Hypo{\tm{r}{r1}\vdash \tm{r}{r2}}
\Infer2{\tm{p}{p7},\tm{r}{r3}\vdash \tm{p}{p8}\land \tm{r}{r4}}
\Infer2{\tm{p}{p9},\tm{q}{q5}\lor \tm{r}{r5}\vdash \tm{p}{p10}\land \tm{q}{q6},\tm{p}{p11}\land \tm{r}{r6}}
\Infer1{\tm{p}{p12},\tm{q}{q7}\lor \tm{r}{r7}\vdash (\tm{p}{p13}\land \tm{q}{q8})\lor(\tm{p}{p14}\land \tm{r}{r8})}
\Infer1{\tm{p}{p15}\land(\tm{q}{q9}\lor \tm{r}{r9})\vdash(\tm{p}{p16}\land \tm{q}{q10})\lor(\tm{p}{p17}\land \tm{r}{r10})}
\end{prooftree}
\]
\begin{tikzpicture}[flowgraph,bend angle=60]
%%% p links
\path (p1) edge[bend left] (p2);
\path (p3) edge[bend left] (p4);
\draw (p15) -- (p12); \draw (p12) -- (p9);
\draw (p9) -- (p5);   \draw (p9) -- (p7);
\draw (p7) -- (p3);   \draw (p5) -- (p1);
\draw (p2) -- (p6);   \draw (p6) -- (p10);
\draw (p10) -- (p13); \draw (p13) -- (p16);
\draw (p4) -- (p8);   \draw (p8) -- (p11);
\draw (p11) -- (p14); \draw (p14) -- (p17);
%%% q links
\begin{scope}[Blue]
\path (q1) edge[bend left] (q2);
\draw (q3) -- (q1); \draw (q5) -- (q3);
\draw (q7) -- (q5); \draw (q9) -- (q7);
\draw (q2) -- (q4); \draw (q4) -- (q6);
\draw (q6) -- (q8); \draw (q8) -- (q10);
\end{scope}
%%% r links
\begin{scope}[Green]
\path (r1) edge[bend left] (r2);
\draw (r3) -- (r1); \draw (r5) -- (r3);
\draw (r7) -- (r5); \draw (r9) -- (r7);
\draw (r2) -- (r4); \draw (r4) -- (r6);
\draw (r6) -- (r8); \draw (r8) -- (r10);
\end{scope}
\end{tikzpicture}
\begin{Verbatim}
\[
\begin{prooftree}
\Hypo{\tm{p}{p1}\vdash \tm{p}{p2}}
\Hypo{\tm{q}{q1}\vdash \tm{q}{q2}}
\Infer2{\tm{p}{p5},\tm{q}{q3}\vdash \tm{p}{p6}\land \tm{q}{q4}}
\Hypo{\tm{p}{p3}\vdash \tm{p}{p4}}
\Hypo{\tm{r}{r1}\vdash \tm{r}{r2}}
\Infer2{\tm{p}{p7},\tm{r}{r3}\vdash \tm{p}{p8}\land \tm{r}{r4}}
\Infer2{\tm{p}{p9},\tm{q}{q5}\lor \tm{r}{r5}
\vdash \tm{p}{p10}\land \tm{q}{q6},\tm{p}{p11}\land \tm{r}{r6}}
\Infer1{\tm{p}{p12},\tm{q}{q7}\lor \tm{r}{r7}\vdash 
(\tm{p}{p13}\land \tm{q}{q8})\lor(\tm{p}{p14}\land \tm{r}{r8})}
\Infer1{\tm{p}{p15}\land(\tm{q}{q9}\lor \tm{r}{r9})
\vdash(\tm{p}{p16}\land \tm{q}{q10})\lor(\tm{p}{p17}\land \tm{r}{r10})}
\end{prooftree}
\]
\begin{tikzpicture}[flowgraph,bend angle=60]
%%%
\path (p1) edge[bend left] (p2);
\path (p3) edge[bend left] (p4);
\draw (p15) -- (p12); \draw (p12) -- (p9);
\draw (p9) -- (p5);   \draw (p9) -- (p7);
\draw (p7) -- (p3);   \draw (p5) -- (p1);
\draw (p2) -- (p6);   \draw (p6) -- (p10);
\draw (p10) -- (p13); \draw (p13) -- (p16);
\draw (p4) -- (p8);   \draw (p8) -- (p11);
\draw (p11) -- (p14); \draw (p14) -- (p17);
%%%
\begin{scope}[Blue]
\path (q1) edge[bend left] (q2);
\draw (q3) -- (q1); \draw (q5) -- (q3);
\draw (q7) -- (q5); \draw (q9) -- (q7);
\draw (q2) -- (q4); \draw (q4) -- (q6);
\draw (q6) -- (q8); \draw (q8) -- (q10);
\end{scope}
%%%
\begin{scope}[Green]
\path (r1) edge[bend left] (r2);
\draw (r3) -- (r1); \draw (r5) -- (r3);
\draw (r7) -- (r5); \draw (r9) -- (r7);
\draw (r2) -- (r4); \draw (r4) -- (r6);
\draw (r6) -- (r8); \draw (r8) -- (r10);
\end{scope}
\end{tikzpicture}
\end{Verbatim}
Again, the thicket of arrows in over an entire sequent derivation makes a flow graph a little hard to scan. You can reuse the code for the derivation and draw a flow graph on its \emph{endsequent}.
\[
\begin{prooftree}
\Hypo{\tm{p}{p1}\vdash \tm{p}{p2}}
\Hypo{\tm{q}{q1}\vdash \tm{q}{q2}}
\Infer2{\tm{p}{p5},\tm{q}{q3}\vdash \tm{p}{p6}\land \tm{q}{q4}}
\Hypo{\tm{p}{p3}\vdash \tm{p}{p4}}
\Hypo{\tm{r}{r1}\vdash \tm{r}{r2}}
\Infer2{\tm{p}{p7},\tm{r}{r3}\vdash \tm{p}{p8}\land \tm{r}{r4}}
\Infer2{\tm{p}{p9},\tm{q}{q5}\lor \tm{r}{r5}\vdash \tm{p}{p10}\land \tm{q}{q6},\tm{p}{p11}\land \tm{r}{r6}}
\Infer1{\tm{p}{p12},\tm{q}{q7}\lor \tm{r}{r7}\vdash (\tm{p}{p13}\land \tm{q}{q8})\lor(\tm{p}{p14}\land \tm{r}{r8})}
\Infer1{\tm{p}{p15}\land(\tm{q}{q9}\lor \tm{r}{r9})\vdash(\tm{p}{p16}\land \tm{q}{q10})\lor(\tm{p}{p17}\land \tm{r}{r10})}
\end{prooftree}
\]
\begin{tikzpicture}[flowgraph,bend angle=45]
\path (p15) edge[bend right] (p16);
\path (p15) edge[bend right] (p17);
\path (q9)  edge[bend right, Blue] (q10);
\path (r9)  edge[bend right, Green] (r10);
\end{tikzpicture}
\begin{Verbatim}
\[
\begin{prooftree}
\Hypo{\tm{p}{p1}\vdash \tm{p}{p2}}
\Hypo{\tm{q}{q1}\vdash \tm{q}{q2}}
\Infer2{\tm{p}{p5},\tm{q}{q3}\vdash \tm{p}{p6}\land \tm{q}{q4}}
\Hypo{\tm{p}{p3}\vdash \tm{p}{p4}}
\Hypo{\tm{r}{r1}\vdash \tm{r}{r2}}
\Infer2{\tm{p}{p7},\tm{r}{r3}\vdash \tm{p}{p8}\land \tm{r}{r4}}
\Infer2{\tm{p}{p9},\tm{q}{q5}\lor \tm{r}{r5}
\vdash \tm{p}{p10}\land \tm{q}{q6},\tm{p}{p11}\land \tm{r}{r6}}
\Infer1{\tm{p}{p12},\tm{q}{q7}\lor \tm{r}{r7}\vdash 
(\tm{p}{p13}\land \tm{q}{q8})\lor(\tm{p}{p14}\land \tm{r}{r8})}
\Infer1{\tm{p}{p15}\land(\tm{q}{q9}\lor \tm{r}{r9})
\vdash(\tm{p}{p16}\land \tm{q}{q10})\lor(\tm{p}{p17}\land \tm{r}{r10})}
\end{prooftree}
\]
\begin{tikzpicture}[flowgraph,bend angle=45]
\path (p15) edge[bend right] (p16);
\path (p15) edge[bend right] (p17);
\path (q9)  edge[bend right, Blue] (q10);
\path (r9)  edge[bend right, Green] (r10);
\end{tikzpicture}
\end{Verbatim}

\end{document}
